\capitulo{2}{Objetivos del proyecto}


A continuación, se detallan los diferentes objetivos que han motivado la
realización del proyecto.

\section{Objetivos generales}\label{objetivos-generales}

\begin{itemize}
	\tightlist
	\item
	Desarrollar una aplicación de escritorio para poder visualizar las interacciones realizadas en el moodle de la Universidad de Burgos, por los diferentes usuarios, en las distintas asignaturas.
	\item
	Facilitar la interpretación de los datos recogidos mediante
	representaciones gráficas.
	\item
	Aportar información extra a los profesores de la asignatura, para saber la interacción que tienen sus alumnos con ella y otros usuarios como puede ser el administrador de moodle.
	
\end{itemize}

\section{Objetivos técnicos}\label{objetivos-tecnicos}

\begin{itemize}
	\tightlist
	\item
	Desarrollar aplicación de escritorio con Java FX.
	\item
	Desarrollar lógica de programación en entorno Java 8
	\item
	Utilización de API WebService de moodle para la obtención de diferentes datos.
	\item
	Parseo de documentos csv.
	\item
	Utilizar Git como sistema de control de versiones distribuido junto
	con la plataforma GitHub.
	\item
	Aplicar la metodología ágil Scrum en el desarrollo del software.
	\item
	Realizar test unitarios, de integración y de interfaz. SADFAFADSFSDD     REVISAR"""""""""""""""""""""
	\item
	Utilizar ZenHub como herramienta de gestión de proyectos.
	\item
	Utilizar Mendeley para almacenamiento de bibliografía.
\end{itemize}

\section{Objetivos personales}\label{objetivos-personales}

\begin{itemize}
	\tightlist
	\item
	Utilizar librería chart.js para generar los gráficos.
	\item
	Utilizar Gradle como herramienta para automatizar el proceso de
	construcción de software.
	\item
	Hacer uso de herramientas de integración continua como Travis y
	SonarQube en el repositorio.
	\item
	Aumentar los tipos de parseo de documentos.
	
\end{itemize}
