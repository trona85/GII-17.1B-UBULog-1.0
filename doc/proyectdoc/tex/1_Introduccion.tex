\capitulo{1}{Introducción}

Los profesores, hoy en día, utilizan plataformas para poder facilitar su trabajo, proporcionando recursos, haciendo exámenes o teses, de manera on-line. No es que se quede anticuada la forma tradicional de impartir clases, si no que evoluciona, para facilitar el trabajo a ellos mismos y a los alumnos.
La más utilizada y a la vez la que estamos utilizando en la Universidad de Burgos es Moodle, en concreto la versión 3.3.

\imagen{citymoodle}{Estadística de países que más utilizan Moodle.}{0.7}

\imagen{cytiremoodle}{Datos de uso de Moodle.}{0.7}

\imagen{versionmoodle}{Versiones más utilizadas.}{0.7}

Lo que queremos conseguir aprovechando el potencial de Moodle, es, una aplicación de escritorio, en la que podremos pasar el log correspondiente, que nos lo generan los registros de Moodle en cada asignatura y el profesor puede acceder a ellos sin ningún problema, para facilitarle su interpretación, con un muestreo de los datos en formato de gráficas, filtrando por usuarios de esa asignatura y los tipos de evento realizado. 
Con esto podremos permitir al profesor un control de su asignatura, casi completo. Por ejemplo, podrá saber cuántas veces un alumno ha entrado en calificaciones, ha subido un trabajo, hasta si un administrador ha estado visitando su asignatura o creado nuevos elementos en ella.


\section{Estructura de la memoria}\label{estructura-de-la-memoria}

La memoria sigue la siguiente estructura:

\begin{itemize}
	\tightlist
	\item
	\textbf{Introducción:} breve descripción del problema a resolver y la
	solución propuesta. Estructura de la memoria y listado de materiales
	adjuntos.
	\item
	\textbf{Objetivos del proyecto:} exposición de los objetivos que
	persigue el proyecto.
	\item
	\textbf{Conceptos teóricos:} breve explicación de los conceptos
	teóricos clave para la comprensión de la solución propuesta.
	\item
	\textbf{Técnicas y herramientas:} listado de técnicas metodológicas y
	herramientas utilizadas para gestión y desarrollo del proyecto.
	\item
	\textbf{Aspectos relevantes del desarrollo:} exposición de aspectos
	destacables que tuvieron lugar durante la realización del proyecto.
	\item
	\textbf{Trabajos relacionados:} Resumen de aplicaciones relacionadas con nuestro proyecto.
	\item
	\textbf{Conclusiones y líneas de trabajo futuras:} conclusiones
	obtenidas tras la realización del proyecto y posibilidades de mejora o
	expansión de la solución aportada.
	
\end{itemize}

Junto a la memoria se proporcionan los siguientes anexos:

\begin{itemize}
	\tightlist
	\item
	\textbf{Plan de Proyecto Software:} análisis de la viabilidad del proyecto y su planificación.
	\item
	\textbf{Especificación de Requisitos:} exposición de los requisitos que debería de tener nuestra aplicación.
	\item
	\textbf{Especificación de diseño:} exposición del diseño de la aplicación.
	\item
	\textbf{Doc. técnica de programación:} aspectos relevantes del código fuente.
	\item
	\textbf{Documentación de usuario:} guía detallada del manejo de la aplicación.
\end{itemize}

\section{Materiales adjuntos}\label{materiales-adjuntos}

Los materiales que se adjuntan con la memoria son: 

\begin{itemize}
	\tightlist
	\item
	Aplicación UBULog.
	\item
	Ejecutable Linux UBULog.sh.
	\item
	Ejecutable Windows UBULog.bat.	
	\item
	JavaDoc.
	\item
	Código fuente.
	\item
	Anexos.
\end{itemize}

Además, los siguientes recursos están accesibles a través de internet:

\begin{itemize}
	\tightlist
	\item
	\href{https://github.com/trona85/GII-17.1B-UBULog-1.0}{Código público en el repositorio de GitHub.}
	\item
	\href{https://sonarcloud.io/dashboard?id=GII-17.1B-UBULog-1.0}{Análisis de la calidad del código con SonarQube.}
	\item
	\href{https://travis-ci.org/trona85/GII-17.1B-UBULog-1.0/}{Integración continua del proyecto con Travis CI.}
	
\end{itemize}



