\apendice{Documentación de usuario}

\section{Introducción}

En este manual detallamos los requisitos necesarios y el manejo de la aplicación, para ayudar al usuario a su comprensión. No es necesaria la instalación, ya que es un ejecutable.

\section{Requisitos de usuarios}

Los requisitos mínimos para poder hacer uso de la aplicación son:

\begin{itemize}
	\tightlist
	\item
	Disponer del ejecutable de la aplicación, UBULog.jar.
	\item
	Disponer de un ordenador con conexión a Internet.
	\item
	Tener instalado Java 8 en el equipo que se vaya a ejecutar la aplicación.
	\item
	El usuario debe tener una cuenta de UBUVirtual activa con el rol de profesor.
\end{itemize}

%\section{Instalación}

\section{Manual del usuario}

\subsection{Ejecución}

Deberemos hacer doble clic en el ejecutable, si estamos en un entorno Linux UBULog.sh y si estamos en entorno Windows UBULog.bat.

En el caso que estemos con un S.O. Linux y queramos ejecutar el .jar directamente, desde la consola deberemos abrirla, movernos a la ubicación del ejecutable y poner ``java -jar nombreaplicación.jar''. En la siguiente captura mostramos el proceso.

\imagen{execut}{Ejecución UBULog.}{0.9}

IMPORTANTE: Se ha detectado que la librería Java FX da errores con algunas versiones de Java 8 cuando intenta mostrar las tablas de log y el gráfico, hay que tener instalado las ultimas versiones.

\subsection{Descargar el log manualmente}

Para descargar el log manualmente deberemos ir a UBUVirtual, iniciar sesión y irnos a la asignatura deseada. Una vez allí en la parte de administración de la asignatura deberemos pinchar en ``Registros''.

\imagen{administracion}{Administración de la asignatura.}{0.5}

\newpage Una vez pinchemos en registro, nos llevará a la siguiente ventana:

\imagen{filtross}{Filtros log.}{0.9}

Ponemos las opciones que deseemos y pinchamos en ``Conseguir estos registros''.

\imagen{filtrosscons}{Registros.}{0.9}

Despues en la parte inferior tendremos un botón que nos pondrá ``Descargar''.

\imagen{logsdescc}{Descarga de logs.}{0.9}

Una vez descargado ya lo podremos utilizar en nuestra aplicación.


\subsection{Login}


Una vez ejecutada la aplicación nos saldrá la ventana para iniciar sesión en nuestra aplicación.

\imagen{login}{Login UBULog.}{0.6}

En esta ventana observamos tres campos que deberemos rellenar:

\begin{itemize}
	\tightlist
	\item
	\textbf{Usuario:} Correo de la personal UBU.
	\item
	\textbf{Contraseña:} Contraseña utilizada para iniciar sesión en UBUVirtual.
	\item
	\textbf{Host:} Dirección web de la plataforma a donde nos queremos conectar. Por defecto está https://ubuvirtual.ubu.es/.
\end{itemize}

Una vez introduzcamos los campos tenemos dos opciones:

\begin{itemize}
	\tightlist
	\item
	\textbf{Clicar borrar:} Se borrarán los campos.
	\item
	\textbf{Clicar entrar:} Pueden suceder 2 cosas:
	\begin{itemize}
		\tightlist
		\item
		\textbf{Datos incorrectos:} Se borrarán los datos e indicará al usuario que el usuario o la contraseña es errónea.
		\item
		\textbf{Datos correctos:} Se nos abrirá la pantalla de bienvenida.
	\end{itemize}
\end{itemize}

\subsection{Pantalla de bienvenida}


Cuando iniciemos sesión en nuestra aplicación llegaremos a la pantalla de bienvenida, en ella, nos aparecerá la lista de asignaturas en las que estamos matriculado. Deberemos seleccionar una en la que estemos matriculados con el rol de profesor.

\imagen{bienvenida}{Pantalla de bienvenida.}{0.9}

Cuando hagamos clic en entrar pueden pasar dos cosas:

\begin{itemize}
	\tightlist
	\item
	\textbf{Si NO hemos seleccionado asignatura:} La aplicación nos informa que debemos seleccionar una asignatura.
	\item
	\textbf{Si hemos seleccionado asignatura:} Nos llevará a la ventana principal de la aplicación.
	
\end{itemize}
\newpage
\subsection{Pantalla principal}

\imagen{ubulog}{Pantalla principal.}{0.9}

\subsubsection{Visualización de participantes}

En esta parte de la aplicación veremos los participantes de la asignatura (matriculados, desconocidos, sistema y administrador), nos dirá el número de participantes y tendremos tres filtros:


\begin{itemize}
	\tightlist
	\item
	\textbf{Filtro participantes:} Es un filtro de tipo texto que filtrará por el nombre de los participantes.
	\item
	\textbf{Grupos:} Filtrará por los grupos a los que pertenece los participantes.
	\item
	\textbf{Rol:} Filtrará por los roles a los que pertenece los participantes.
	
\end{itemize}


\imagen{participantes}{Visualización de participantes.}{0.6}

Esta sección de la aplicación estará deshabilitada hasta que se cargue el log. Una vez cargado podremos interactuar con él para mostrar los datos en la gráfica y en la tabla de log.
\newpage
\subsubsection{Visualización de eventos}

En esta parte de la aplicación veremos los eventos que se han realizado a lo largo de la asignatura y un filtro de texto para filtrar por el nombre del evento.


\imagen{evento}{Visualización de eventos.}{0.5}

Esta sección de la aplicación estará deshabilitada hasta que se cargue el log. Una vez cargado podremos interactuar con él para mostrar los datos en la gráfica y en la tabla de log.

\subsubsection{Gráfica}

En esta parte de la aplicación podremos ver los datos seleccionados de forma gráfica, en esta ventana, también tenemos la posibilidad de cambiar el tipo de gráfico.

\imagen{grafica}{Gráfica.}{0.9}
\newpage
\subsubsection{Tabla logs}

En esta parte de la aplicación podremos ver los log correspondientes a los datos seleccionados en forma de tabla, en esta ventana, también tenemos la posibilidad de filtrar los datos con los filtros que se encuentra en la parte superior, cada uno corresponde a una columna de la tabla y no diferencian entre minúsculas y mayúsculas.

\imagen{tablelog}{Tabla logs.}{0.9}

\subsubsection{Botones inferiores}

En esta parte de la aplicación podremos clicar los siguientes botones y su comportamiento es el siguiente:

\begin{itemize}
	\tightlist
	\item
	\textbf{Generar gráfica a partir de filtros de tabla:} Este botón permanecerá deshabilitado hasta que se cargue el log. De lo que se encarga es de coger los log de la tabla resultante sin filtrar o si se han aplicado filtros y generar un nuevo gráfico con los nuevos datos.
	\item
	\textbf{Exportar gráfico:} Guarda el gráfico en una imagen.
	\item
	\textbf{Cargar documento online:} Carga los datos del log de forma automatizada. Este proceso puede tardar varios minutos.
	\item
	\textbf{Cargar documento local:} Habiendo descargado el log de la asignatura en CSV, podemos pasárselo a la aplicación directamente y cargar los datos. Este proceso es más eficiente que el punto anterior.
\newpage	\item
	\textbf{Salir:} Cierra la aplicación.
	
	
\end{itemize}

\imagen{botonesinferiores}{Botones inferiores.}{0.9}

El botón ``Generar gráfica a partir de filtros de texto'' estará deshabilitado hasta que se cargue el log. Una vez cargado podremos interactuar con él para mostrar los datos en la gráfica.

\subsubsection{Información del usuario}

En esta parte de la aplicación podremos ver el nombre, foto y curso del usuario y el \emph{host} en el que se ha conectado.

\imagen{userop}{Información del usuario.}{0.9}

\subsubsection{Menú}

\imagen{menu}{Menú.}{0.9}

En este menú podremos hacer las siguientes acciones:

\begin{itemize}
	\tightlist
	\item
	\textbf{Archivo:}
	\begin{itemize}
		\tightlist
		\item
		\textbf{Cambiar asignatura:} Vuelve a la pantalla de bienvenida, que muestra la lista de cursos matriculados del usuario.
		\item
		\textbf{Guardar gráfico como...:} Abre una ventana para que el usuario guarde el gráfico en formato de imagen (.jpg o .png).
		\item
		\textbf{Cerrar sesión:} Vuelve a la pantalla de inicio de sesión.
		\item
		\textbf{Salir:} Cierra la aplicación.
	\end{itemize}
	\item
	\textbf{Editar:}
	\begin{itemize}
		\tightlist
		\item
		\textbf{Borrar selección:} Borra las selecciones de los participantes y eventos que estén seleccionados y recarga el gráfico y la tabla de log.
	\end{itemize}
	\item
	\textbf{Ayuda:}
	\begin{itemize}
		\tightlist
		\item
		\textbf{Acerca de UBULog:} Abre en el navegador un enlace al repositorio del proyecto.
	\end{itemize}
\end{itemize}










