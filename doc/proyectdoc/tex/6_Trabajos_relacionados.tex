\capitulo{6}{Trabajos relacionados}

\section{UBUGrades 1.0}\label{ubugrades--10}

Esta aplicación es el TFG de Claudia Martinez Herrero \cite{claudia} del que se ha partido para hacer nuestra aplicación.

En esta aplicación se recolectan los datos del Web Service de Moodle para sacar gráficas con las notas de los alumnos de la plataforma UBUVirtual.

\newpage\section{Heatmap}\label{heatmap}

\href{https://moodle.org/plugins/block_heatmap}{\emph{Heatmap}} es un plugin de Moodle que nos permite hacer mapas de calor dependiendo del numero de interacciones que hayan hecho los participantes.

Este plugin está inspirado en \emph{Moodle Activity Viewer}.

\imagen{heatmap}{Plugin Heatmap}{0.7}

Aunque nuestra aplicación no hace mapas de calor de las actividades realizadas, podemos generar un gráfico y ver más detalladamente las iteraciones que hace el usuario en las diferentes actividades y saber cuando lo ha hecho y cuantas veces, también podemos ver todas las interacciones en la tabla de log de una manera mas específica.


\newpage\section{ Moodle Activity Viewer}\label{moodle-activity-viewer}

\href{https://damos.world/2013/08/30/the-moodle-activity-viewer-mav-heatmaps-of-student-activity/}{ \emph{Moodle Activity Viewer}} es un plugin de Moodle que nos permite hacer mapas de calor de las actividades de la asignatura dependiendo de las interacciones que tienen en ellas los participantes.

\imagen{mav}{Plugin Moodle Activity Viewer}{0.9}

Con este plugin de Moodle ocurre lo mismo que con el plugin mencionado anteriormente (Heatmap).

