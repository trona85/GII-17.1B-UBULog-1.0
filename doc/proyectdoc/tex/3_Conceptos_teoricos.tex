\capitulo{3}{Conceptos teóricos}

CITAR TFG CLAUDIA APARTADO CONCEPTOS TEORICOS PAG 6 A 13.

NUEVO CONTENIDO ......

\section{API Rest}\label{api-rest}

\section{Token}\label{token}
\section{Json}\label{json}
\section{Web Scraping}\label{web-scraping}

Web Scraping es una manera de interaccionar con una web y automatizar las interacciones. Su uso habitual es para  el testeo de webs, probar campos, botones, etc.... Pero también podemos utilizarlo para el tratamiento de datos. Hay veces que la web no proporciona una api rest o un archivo con el cual podamos tratar ciertos datos para hacer "x" cosa. En nuestro caso, lo tenemos que utilizar para la descarga automática de un log, en el que si no estamos logeados no podríamos descargar.

El usuario podría ir al registro sin ningún problema y descargarlo, pero también nos interesa, que si el usuario no sabe donde esta el registro o le resulta dificultoso el proceso de descargarlo manualmente, podamos automatizarlo para el.
 
Lo que hemos hecho es automatizar el proceso que haría el usuario de forma manual. 
 \begin{enumerate}
 	\item 
 	Nos iremos a la url de logueo.
 	\item 
 	Rellenamos usuario y contraseña y inmediatamente hacemos click en aceptar.
 	\item 
 	Construimos la url para ir al registro con los filtros para que salgan todos, se puede poner en la url porque moodle, esa búsqueda, hace una petición GET.
 	\item 
 	Una vez estemos en la ventana con todos los log del registro hacemos click en descargar, como por defecto el archivo descarga es csv no consideramos ese movimiento.
 	\item 
 	Recogemos la respuesta que nos da la web y la convertimos en String, en este paso ya tenemos todo el registro almacenado en nuestra aplicación.
 \end{enumerate}
 
Aunque en nuestra aplicación ya nos logeamos al principio, para este proceso hay que volver a logearse, ya que, es token que nos proporciona el web service de moodle, no es el mismo.

Es importante destacar, que este proceso, para el usuario, es transparente. Lo único que va a notar, es que, le cuesta cargar el log mas. En el equipo que se ha desarrollado el proyecto un log de 100 registros es imperceptible pero hacer que se lo descargue automáticamente, todo el proceso son unos 6 segundos aproximadamente.
 