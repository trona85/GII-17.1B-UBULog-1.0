\apendice{Documentación técnica de programación}

\section{Introducción}

En este anexo describe la documentación técnica de programación,
incluyendo la instalación del entorno de desarrollo, la estructura de la
aplicación, su compilación, ejecución, la configuración de los diferentes servicios y las pruebas del sistema.

\section{Estructura de directorios}

El repositorio del proyecto podemos observar la siguiente estructura:

\begin{itemize}
	\tightlist
	\item
	\texttt{/}: contiene los ficheros de configuración de Gradle \cite{gradle}, de los servicios de integración continua, en nuestro caso Travis-CI \cite{web:travis}, el fichero README , la copia de la
	licencia y un fichero .xml que es el ANT para la creación del .jar.
	\item
	\texttt{/src/main/java/}: Encontraremos toda la lógica de la aplicación.
	\item
	\texttt{/src/test/java/}: El código para testear la aplicación.
	\item
	\texttt{/resources/}: Recursos de nuestra aplicación.
	\item
	\texttt{/resources/css/}: Estilos de la aplicación.
	\item
	\texttt{/resources/img/}: Imágenes de la aplicación.
	\item
	\texttt{/lib/}: Librerías de la aplicación.
	\item
	\texttt{/doc/}: documentación del proyecto.
	\item
	\texttt{/doc/javadoc/}: documentación \emph{javadoc}.
	\item
	\texttt{/docs/proyectdoc/}: documentación en formato \LaTeX.
\end{itemize}

\section{Manual del programador}

En esta sección explicaremos como montar el entorno de desarrollo, descargar el código fuente del proyecto, compilarlo, ejecutarlo y exportarlo. hay que tener en cuenta que este manual esta orientado a sistemas Linux como lo es Ubuntu 16.04 que es donde se ha desarrollado el proyecto, aunque también se podría desarrollar en S.O. Windows.

Para la instalación en sistemas Windows hay que leer la documentación de Claudia Martínez Herrero \cite{claudia:anexo} referente al manual del programador.

\subsection{Entorno de desarrollo}\label{entorno-de-desarrollo}

Para trabajar con el proyecto se necesita tener instalados los
siguientes programas y dependencias:

\begin{itemize}
	\tightlist
	\item
	Java JDK 8.
	\item
	Git.
	\item
	Moodle 3.3.
	\item
	Eclipse.
	\item
	Plugin Gradle para Eclipse.
\end{itemize}

A continuación, se indica como instalar y configurar correctamente cada
uno de ellos.

\subsubsection{Java JDK 8}\label{java-jdk-8}

En esta instalación es importante descargar el JDK 8 de oracle ya que, otro jdk como puede ser el que viene por defecto en las librerías de Ubuntu no te proporcionan la Librería Java FX.

Deberemos abrir la consola y poner los siguientes comandos:

\imagen{jdk}{Instalacion JDK 8 en Linux.}

\subsubsection{Git}\label{git}

Necesitamos git para la gestión del repositorio. Git nos permitirá clonar el proyecto en nuestro equipo, subir cambios, comprobar diferencias entre commit, etc.

\imagen{git}{Instalacion Git.}

\subsubsection{Moodle 3.3}\label{moodle}

Para la instalación de Moodle hacemos referencia a los anexos de Claudia Martínez Herrero \cite{claudia:anexo} referente a la instalación de moodle en Windows ya que es idéntica.

\subsubsection{Eclipse}\label{git}

Eclipse es el entorno de desarrollo seleccionado para este proyecto. Si el programador desea utilizar otro, no tendría ningún problema.

Para la instalación deberemos descargar \href{http://www.eclipse.org/downloads/}{Eclipse} en nuestro equipo. Lo descomprimimos y ejecutamos el archivo eclipse-inst, los pasos siguientes es igual que en Windows, se abrirá el gestor de instalación y solo hay que seguir las instrucciones.

\subsubsection{Gadle}\label{git}

La mayoría de los IDE de desarrollo tienen integrado Gradle, como es el caso de Eclipse y no se necesita ningún tipo de instalación.

En el caso de que no tenga, podemos instalarlo en nuestra maquina personal. Lo que deberemos hacer es:

\imagen{gradle}{Instalación gradle.}

\section{Compilación, instalación y ejecución del proyecto}

\subsubsection{Obtención del código fuente}\label{obtención-del-código-fuente}

Una vez hemos instalado todo lo necesario, deberemos clonar nuestro proyecto.

Deberemos entrar al repositorio \href{https://github.com/trona85/GII-17.1B-UBULog-1.0}{GitHub} y en la parte que pone Clone whith HTTPS, copiarnos la URL.

\imagen{githubclone}{Clonar proyecto con Git 1.}

Abrimos la consola y nos vamos a nuestro workspace. Una vez ahí, ponemos el siguiente comando de git con la URL copiada anteriormente.

\imagen{gitclone}{Clonar proyecto con Git 2.}

Nos descargara el código de la aplicación y ya lo tendremos en nuestro equipo para poder trabajar con el.

\subsubsection{Importar proyecto en Eclipse}\label{importar-proyecto-en-eclipse}

Para importar nuestro proyecto deberemos abrir Eclipse. Una vez abierto deberemos:

\begin{enumerate}
	\tightlist
	\item
	Vamos a File-->import...
	\item
	Nos saldrá una ventana en la que deberemos elegir la opción de gradle.
	
	\imagen{importgradle}{Importar proyecto Gradle.}
	
	\item
	Nos saldra una ventana de información, después de su lectura se da a next.
	
		\imagen{welcomegradle}{Información Gradle.}
	\item
	En esta ventana nos ubicamos en la raíz del proyecto que queremos importar, igual que en la siguiente imagen, después pinchamos en finalizar.
	
	\imagen{raizgradle}{Cargar proyecto Gradle.}
	
	\item
	Gradle empezara a descargar todas las dependencias del proyecto, esto puede tardar unos minutos. Cuando termine ya tendremos nuestro proyecto preparado para el desarrollo.
		
	\imagen{dependeciesgradle}{Cargar proyecto Gradle.}
	
	
\end{enumerate}

\subsubsection{Build.gradle}\label{buildgradle}

Build.gradle es el archivo donde debemos poner el tipo de aplicación que es, la versión y las dependencias necesarias de nuestro proyecto. \cite{gradle}

\subsubsection{Travis.yml}\label{buildgradle}

Travis.yml es el archivo donde tenemos la configuración para la integración continua y el análisis de la calidad del código con sonarqube. \cite{web:travis} \cite{Web:sonarqube}

\subsubsection{Generar .jar}\label{generar-jar}

Para generar el .jar se ha creado un .xml en el proyecto que es un ANT, en el se tienen todas las dependencias, .class y recursos necesarios para la ejecución. Para generar el jar deberemos:

\begin{enumerate}
	\tightlist
	\item
	Abrimos UBULog.xml en eclipse.
	\item
	La ventana de Eclipse que esta en la parte derecha, que pone OutLine tendremos una opción que es create runnable.
	
	\imagen{antecl}{Crear .jar.}
	
	Con el botón secundario pinchamos y ejecutamos el ANT Builder para generar el jar, en la consola del Eclipse nos saldrá esto.
	
	\imagen{antcreate}{Jar creado.}
	
	Una vez visto el resultado de la captura anterior, querrá decir que ya hemos generado el .jar.
	
\end{enumerate}

%\section{Pruebas del sistema}
