\capitulo{7}{Conclusiones y Líneas de trabajo futuras}

En esta última sección expondremos las conclusiones del trabajo realizado y las líneas futuras que se podrán seguir.

\section{Conclusiones}\label{conclusiones}

En la finalización del proyecto podemos sacar las siguientes conclusiones.

\begin{itemize}
	\tightlist
	\item
	El objetivo general del proyecto se ha cumplido satisfactoriamente.
	Ahora los profesores cuentan con una aplicación para poder ver las interacciones de los diferentes usuarios, y hacer comprobaciones entre equipos de alumnos para saber quién ha interaccionado con UBUVirtual y quién no, para tener un análisis más detallado.
	\item
	El proyecto ha sido desarrollado con un claro pensamiento de mantenibilidad. Esto ha conllevado que, al hacer algún cambio de implementación, el tiempo invertido en ello era mínimo. Esta última consideración está ligada también a una nueva implementación.
	\item
	Se ha observado la importancia de las herramientas como Travis o SonarQube para el desarrollo.
	\item
	El editor de \LaTeX\ ha sido de gran ayuda en la documentación, aunque sea de mayor costumbre utilizar Word o Writer, ha sido bueno el aprendizaje de esta tecnología para futuros usos en las empresas privadas.
	\item
	Hay que tener cuidado con la documentación que existe en internet, aunque sea la oficial, en alguna librería utilizada la documentación conducía a errores de implementación y hay que observar el propio código.
	\item
	Al utilizar la metodología ágil Scrum, se ha desarrollado el proyecto de la forma más profesional posible, sufriendo la presión de las entregas, lo que será habitual en el entorno laboral.
	\item
	Es complicado calcular la duración de los \emph{issue}, se puede decir, que he sido optimista en el cálculo, excepto en un par de \emph{issue} los demás se finalizaban antes de lo esperado.
	
\end{itemize}

\section{Líneas de trabajo futuras}\label{luxedneas-de-trabajo-futuras}

\begin{itemize}
	\tightlist
	\item
	Buscar funciones de la API Web Service de Moodle para dar información más concreta al usuario, diciéndole, por ejemplo, información concreta de cuestionarios, preguntas, contenido, etc....
	\item
	Investigar la mejora de tiempos del Web Scripting o incluso buscar una librería mejor para este proceso. El tiempo que tarda en descargar los log es muy grande y cuánto más grande es el log aumenta, de tal forma, que puede llegar a bloquear la aplicación.
	\item
	Como en los log tenemos la IP por donde se han conectado a UBUVirtual, podríamos hacer estadísticas de por donde se conectan las personas o incluso hacer minería con estos datos para saber cuándo se conectan más.
	\item
	Añadir más tipos de gráficos. Hemos incluido 3, pero convendría añadir más tipos ya que hay muchos tipos de usuarios y para diferentes datos habrá diferentes gráficos.
	\item
	Al descargar el log, hay unos select que nos permite elegir ciertas opciones para filtrar los log y después descargar esos log. Habría que implementar un método que nos permita modificar esos campos y descargar un log que en un primer momento ya nos vendría filtrado. Un ejemplo sería, que el usuario solo le interesen los log referentes a Diciembre.
	\item
	Implementar traducciones para tener una aplicación en diferentes idiomas.
	\item
	Ampliar formatos de lectura de registros a .xlsx, .html, .jsom, .odt.
	
\end{itemize}
